\documentclass[12pt]{article}

\usepackage{latexsym}
\usepackage{amssymb}
\usepackage{amsmath}
\usepackage{amsfonts}
\usepackage{xspace}    % takes care of conditional inclusion of space
\usepackage{enumitem}
\usepackage{diagrams}

% setup margins:
\topmargin -1.5cm
\oddsidemargin -0.04cm
\evensidemargin -0.04cm
\textwidth 16.59cm
\textheight 21.94cm

\newcommand{\Z}{\mathbb{Z}}
\newcommand{\zp}{\ensuremath{\Z/p\Z}}
\newcommand{\ppt}{\ensuremath{\mathsf{PPT}}}
\newcommand{\indppt}{\underset{{ }^{\ppt}}{\thickapprox}}
\newcommand{\inv}{^{-1}} %saves a little typing...
\newcommand{\set}[1]{\left\{ #1 \right\}}
\newcommand{\pr}[2]{\Pr_{#1}\left[#2\right]}
\def\rand{\stackrel{{}_{\$}}{\leftarrow}} %saves a little space
\newcommand{\bit}{\set{0,1}}
\newcommand{\gen}[1]{\left\langle #1 \right\rangle}
\newcommand{\poly}{{\mbox{\rm poly}}}
\renewcommand{\to}{\longrightarrow}
%%%%%%%%%%%%%%%%%%%%%%%%%%%%%%%%%%%%%%%%

\begin{document}
\title{CSC I4900 Midterm}
%\author{William E. Skeith III}
\date{}

\maketitle
\noindent
\textbf{Important:} please answer \textbf{only 3} of the following questions.
Answer the three you feel most confident about, and leave everything else
blank.  \textbf{Reminder:} You can use your computer, our lecture notes, our
books (Nigel Smart, Abhi + Rafael) and chat with me on Twitch but please use
\textbf{nothing else}.  Do not look for answers online, and do not communicate
with anyone but me during the exam.  You'll have 150 minutes for the exam.
\newline \newline \newline
\textbf{Name: }\hrulefill
\newline\newline

\begin{center}
\begin{tabular}{|c|c|}
  \hline
  Problem & Score \\
  \hline
  1 &
  \\
  (10 points)& \\
  \hline
  2 &  \\
  (10 points)& \\
  \hline
  3 &    \\
  (10 points)& \\
  \hline
  4 &    \\
  (10 points) & \\
  \hline
  5 &    \\
  (10 points)& \\
  \hline
  6 &    \\
  (10 points)& \\
  \hline
  7 &    \\
  (10 points)& \\
  \hline
  8 &    \\
  (10 points)& \\
  \hline
  9 &    \\
  (10 points)& \\
  \hline
  total: &  \\
  30 points& \\
  \hline
\end{tabular}
\end{center}


\newpage

\begin{enumerate}
  \item Prove that the DDH assumption is \textbf{false} for the group
	\(G=\Z_p^\times\), where \(p\) is a prime.  Take for granted some
	generator \(\gen{g} = \Z_p^\times\).
  \item Suppose that \(\gamma\in \Z_p^\times\) generates a group \(G\) of
	prime order \(q\), and say \(q^2\nmid (p-1)\).  With such parameters, the
	DDH assumption is conjectured to hold.  However, if we were to use this
	group for ElGamal encryption, the plaintext space would be
	\(\gen{\gamma}\) and not all of \(\Z_p^\times\).  Show in fact that
	ElGamal is \textbf{not} IND-CPA secure if we use it to encrypt arbitrary
	messages in \(\Z_p^\times\).  (That is, if we are allowed to encrypt
	values that are not a power of \(\gamma\).) \emph{Hint:} multiplying
	\(h\in\gen{\gamma}\) with \(k\not\in\gen{\gamma}\) results in
	\(hk\not\in\gen{\gamma}\), and if you think about the CRT, you will see
	that it is easy to check whether or not a value is in \(\gen{\gamma}\).
	How could you modify the scheme to encrypt arbitrary bit strings (say of
	length \(\ell < \log_2 q\)) if given a hash function
	\(H:\gen{\gamma}\to\bit^\ell\)?  (You can model the hash function as a
	random oracle.)
  \item Suppose that a prime number \(p\) is of the form \(p = 2^k+1\).
	Describe an efficient algorithm for computing discrete logarithms in
	\(\Z_p\).  That is, an efficient procedure for computing the exponent
	\(x\) when given the value \(y = g^x\mod p\) where \(\gen{g} =
	\Z_p^\times\).  (``Efficient'' means polynomial time in \(k\).)
	\emph{Hint:} you can learn the bits of the exponent with the help of the
	Legendre symbol \(x^{(p-1)/2}\).
  \item Demonstrate that public-key encryption which protects against a
	\emph{computationally unbounded} adversary is not possible to achieve.
	Give as formal an argument as you can.
  \item Describe a protocol for \emph{re-sharing} a secret that has already
	been shared via polynomial interpolation (the Shamir scheme).  The
	protocol should \textbf{not} at any point reconstruct the original secret,
	and the new shares should be distributed identically to ``fresh'' shares.
	You can assume the parties are ``honest but curious'' and will faithfully
	execute the protocol you specify, but might try to learn additional
	information from the communications.
  \item A \(t\)-out-of-\(n\) secret sharing scheme is
	information-theoretically secure if any subset of \(\leq t\) shares
	reveals no information about the secret.  More precisely, for any values
	\(s_1,\dots,s_t\) of a collection of \(t\) shares and for any probability
	distribution \(S\) on the space of secrets, \[\pr{}{S=s \mid
	s_1,\dots,s_t} = \pr{}{S=s}.\]  Show that any information-theoretically
	secure secret sharing scheme which shares secrets from a finite set \(X\)
	using shares from another finite set \(Y\) must satisfy \(|X|\leq |Y|\).
	\emph{Hint:} it's a lot like the proof that you need to use long keys for
	information-theoretically secure encryption.
  \item A function \(f:X\to Y\) is said to be \emph{random-self-reducible} if
	an efficient procedure to compute \(f\) on \emph{random} instances can be
	used to compute \(f\) on \emph{any} instance.  Put another way, the
	existence of an oracle \(\mathcal{O}\) with the following property
	\[\pr{x\rand X}{\mathcal{O}(x) = f(x)} \geq \frac{1}{\poly(\ell)}\]
	(where \(\ell\) represents a security parameter), would imply the
	existence of an efficient\footnote{By this, we mean that the algorithm
	always runs in polynomial time and succeeds in computing \(f(x)\) with all
	but negligible probability (independent of the instance \(x\)!).}
	procedure (which can make calls to \(\mathcal{O}\)) for
	computing  \(f\) on \emph{any} input.  Show how both inverting RSA
	(computing \(m\) from \(x = m^e\mod n\)) and the discrete log problem
	(computing \(a\) from \(x = g^a\mod p\)) are random-self-reducible.
	\emph{Note:} the oracle \(\mathcal{O}\) only ``works'' on uniformly random
	instances!  You can't make any assumptions about what it does for specific
	or arbitrarily chosen inputs.  You only know what it does on average.
  \item A \emph{block cipher} (like AES) applies a pseudorandom permutation to
	a message block based on a key \(k\): \(C_k:\bit^\ell\to\bit^\ell\).  To
	encrypt messages \(m\in \bit^*\) which are longer than \(\ell\) bits, one
	approach (called ``electronic codebook mode'') is to simply apply \(C_k\)
	to each \(\ell\)-bit block of message \(m\), concatenating the results to
	produce the ciphertext.  Prove that this approach is \textbf{not} IND-CPA
	secure.
  \item Let \(n\) be an RSA number (\(n = pq\), where \(p,q\) are distinct
	\(\ell\)-bit primes) and \(y = x^2 \mod n\), for some \(x\in
	\Z_n^\times\).  Suppose that a prover \(P\) who knows \(x\) wishes to
	prove its knowledge of a square root \(x\) this to a verifier \(V\) that
	knows only \(y\).  Consider the following protocol:
    \begin{diagram}[width=5em,height=2em]
    P         & \null                                   & V \\
    \null     &  \rTo^{h = r^2\mod n}                   & \null \\
    \null     &  \lTo^{b\in \bit}                       & \null  \\
	\null     &  \rTo^{z = rx^b}  & \null  \\
    \end{diagram}
	Here, \(r\in_R \Z_n^\times\) is chosen by the prover and \(b\in_R\bit\) is
	chosen by the verifier.  The above is repeated \(k\) times (with
	independent random choices for \(r,b\) in each round) and the verifier
	accepts the proof if and only if in each round \(z^2 = hy^b\).  Show that
	this protocol is zero knowledge for an honest verifier.  Further, show
	that it is actually a proof of knowledge -- if the prover could be rewound
	to answer different values of \(b\) with the same value of \(r\), then the
	secret square root \(x\) could be extracted.

\end{enumerate}

\end{document}
