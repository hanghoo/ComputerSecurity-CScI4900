\documentclass[12pt]{article}
\usepackage{latexsym}
\usepackage{amssymb}
\usepackage{amsmath}
\usepackage{amsfonts}
\usepackage{xspace}    % takes care of conditional inclusion of space
\usepackage{enumitem}

% setup margins:
\topmargin -1.5cm
\oddsidemargin -0.04cm
\evensidemargin -0.04cm
\textwidth 16.59cm
\textheight 21.94cm

\newcommand{\Z}{\mathbb{Z}}
\newcommand{\zp}{\ensuremath{\Z/p\Z}}
\newcommand{\ppt}{\ensuremath{\mathsf{PPT}}}
\newcommand{\indppt}{\underset{{ }^{\ppt}}{\thickapprox}}
\newcommand{\inv}{^{-1}} %saves a little typing...
\newcommand{\set}[1]{\left\{ #1 \right\}}
\newcommand{\pr}[2]{\Pr_{#1}\left[#2\right]}
\def\rand{\stackrel{{}_{\$}}{\leftarrow}} %saves a little space
\newcommand{\bit}{\set{0,1}}
\renewcommand{\to}{\longrightarrow}
\newcommand{\te}{there exists\xspace}
\newcommand{\st}{such that\xspace}
\newcommand{\ie}{\emph{i.e.,}\xspace}
%%%%%%%%%%%%%%%%%%%%%%%%%%%%%%%%%%%%%%%%
%vim syntax hack.
\newif\ifsyntaxhack
\syntaxhackfalse
\ifsyntaxhack
\section*{}
\fi
%%%%%%%%%%%%%%%%%%%


\begin{document}
\title{Homework 1}
\date{}

\maketitle

\section{Part I -- Basic Number Theory} % {{{[[[


\begin{enumerate}
\item Compute \(\varphi(n)\) for \(n=2,5,6,8,12\).

\item Compute:
  \begin{itemize}[noitemsep]
	\item \(2^4 \mod 5\)
	\item \(3^{6} \mod 7\)
    \item \(4^{8}\mod 15\)
    \item \(4^{24}\mod 15\)
	\item \(15^{66} \mod 23\)
	\item \(43^{48} \mod 105\)
  \end{itemize}

\item You should have noticed a pattern in the above computations.  Try to
  generalize what is going on.  Does the result of \(33^{48} \mod 105\)
  contradict your conjecture?  If no, congratulations.  Otherwise, fix your
  generalization.

\item How many solutions does the equation \(7x=14\mod 35\) have in
    \(\Z_{35}\)?

\item How many solutions does the equation \(6x=14\mod 35\) have in
    \(\Z_{35}\)?

\item How many solutions does the equation \(10x=14\mod 35\) have in
    \(\Z_{35}\)?
\item Try more examples and see if you can you generalize what is going on.

\item Find the multiplicative inverse of \(22\) in \(\Z_{35}\).

\item Prove that if \(p\) is prime, \(\varphi(p^\alpha) = p^{\alpha-1}(p-1)\).

\item Using the exercise above and the fact that \((a,b)=1 \implies
  \varphi(ab) = \varphi(a)\varphi(b)\), prove the following formula for
  \(\varphi(n)\) in terms of the factorization \(n =
  \prod_{i=1}^{k}p_i^{\alpha_i}\): \[\varphi(n) = n\prod_{i=1}^{k}
  \left(1-\frac{1}{p_i}\right).\]

\item Find an integer \(n\in\Z^+\) such that \[\frac{n}{\varphi(n)} >
    10.\]

\item  Write a program (say using the GMP library) that takes an integer
  \(\ell\) as input and outputs a random prime number of \(\ell\) bits.  Use
  the output to plot the running time of GMP's \texttt{factor} program against
  the bit length of the inputs (where each input is the product of two equal
  length primes).

\end{enumerate}

\subsection*{Bonus Number Theory Questions}
\noindent\textbf{NOTE: you might find the following a little more challenging.
Don't kill yourself over these.}
\begin{enumerate}

\item For any odd prime integer \(p\), prove that \((p-1)! = -1 \mod p\).

\item Suppose that \(p\) is a prime, and that \(p\equiv 3 \mod 4\).  Show
    that there is no integer \(x\in\zp\) such that \(x^2 = -1\).  (Hint:
    think about Lagrange's theorem.)

\item Show that for any \(c\in\Z^+\) \te\ \(n\in\Z^+\) \st\ \(n/\varphi(n) >
    c\), \ie\ that \(\limsup_{n\rightarrow \infty}n/\varphi(n) = \infty\).

\end{enumerate}

% section Part I -- Basic Number Theory (end) ]]]}}}

\section{Part II -- Security Definitions} % {{{[[[

\begin{enumerate}
  \item Let a \(M\) be a finite set of messages, and let \(S(M)\) denote the
	set of all permutations of \(M\) (all bijective functions \(f:M\rightarrow
	M\)).  We'll assume that if given a description of \(\sigma\in S(M)\),
	both \(\sigma\) and \(\sigma\inv\) are efficiently computable.  Suppose
	\(P\subset S(M)\) is such that \(\forall x,y\in M\), \(\exists \sigma\in
	P\) such that \(\sigma(x) = y\).
	\begin{enumerate}[noitemsep]
	  \item Show that \(|P|\geq |M|\).  (This is easy, but makes sure you've
		parsed the definition.)
	  \item Show that if \(|P| = |M|\), then the following encryption scheme
		is \emph{perfectly secure}, provided you only use it once:
		\begin{itemize}[noitemsep]
		  \item Key generation: select a random \(\sigma\in P\);
		  \item Encryption: \(m\mapsto \sigma(m)\)
		  \item Decryption: \(c\mapsto \sigma\inv(c)\)
		\end{itemize}
	  \item Show that the above is false if \(|M| < |P| < 2|M|\).
	  \item Observe that for any finite group \(G\) and any \(g\in G\), the
		map \(x\mapsto gx\) is a permutation of \(G\).  By viewing \(G\)
		itself as a set of permutations of \(G\) in this way, show that the
		above property is satisfied (with \(M = P = G\)).
	  \item The traditional \texttt{xor} one time pad is a special case of the
		above.  What is the finite group in this case?
	\end{enumerate}
  \item Suppose you want to encrypt a single bit, say via a OTP, but you only
	have access to a \emph{biased} coin for key generation (that is, one
	outcome of the coin might be slightly more probable than the other).  Show
	that if you use a single coin toss for key generation, your scheme will
	\textbf{not} be perfectly secure.  How might you generate a uniformly
	random key (50/50 chance for 0/1) with this coin by flipping it multiple
	times?  (This is a bit tricky!)

  \item Suppose an encryption scheme acts on ascii-formatted plaintext
	messages by permuting the ascii characters.  That is, a message \(m =
	a_1\dots a_n\) would be encrypted as \(a_{\pi(1)}\dots a_{\pi(n)}\) for
	some (possibly randomized) permutation \(\pi\) (the \(a_i\) are the
	characters of the mesage).  Prove such an encryption scheme can never be
	IND-CPA secure.

% \item Can a substitution cipher ever be CPA secure?  Prove your answer.

  \item If an encryption scheme is IND-CPA secure, and if \(D\) is the
	decryption function, how must \(|D\inv(x)|\) relate to the security
	parameter (asymptotically) for any (efficiently computable) \(x\) in the
	message space?  Conclude in particular that a public-key,
	\emph{deterministic} encryption scheme (like vanilla RSA) can never be
	IND-CPA secure.

  \item Suppose a public-key cryptosystem encrypts integers (say, modulo
	another integer \(n\)).  Let \(E,D\) denote the encryption and decryption
	algorithms, respectively.  Show that if this scheme has the property that
	\(D(E(x)E(y))\allowbreak = x+y\) for any messages \(x,y\), then the scheme
	is necessarily vulnerable to a CCA2 attack.

  \item For IND-CPA security, recall that we had two definitions: a ``game-style''
  definition, and the ``semantic'' definition, stating that for all
  distributions \(D\) on the message space \(M\), and for all predicates \(P:
  M\to \bit\), any efficient algorithm that predicts the predicate on input of
  a ciphertext will succeed with probability at most \(l+\epsilon\), where \[l
  = \max_{b\in \bit} \pr{m\rand D}{P(m)=b}\] and where \(\epsilon\) is
  negligible in the security parameter. Show that if an encryption scheme is
  secure according to the game-style definition, then it is secure under the
  semantic definition.  \emph{Note: the converse is also true.  Try to prove
  that as well.  It is a little harder though.}

  \item Consider an encryption scheme \((G,E,D)\) with the following property:
	in addition to the usual key generation algorithm \(G\), there exists an
	algorithm \(\widetilde G\) such that
	\begin{itemize}[noitemsep]
	  \item \(G(1^\lambda)\indppt \widetilde G(1^\lambda)\), and yet,
	  \item for \(\widetilde{\mathsf{pk}}\leftarrow \widetilde G\), it holds
		that for all equal-length messages \(m_0,m_1\), the distributions
		\(E(\widetilde{\mathsf{pk}},m_0), E(\widetilde{\mathsf{pk}},m_1)\) are
		identically distributed.
	\end{itemize}
	\begin{enumerate}[noitemsep]
	  \item Prove that any such cryptosystem must be IND-CPA secure.
	  \item Show how to construct such a cryptosystem based on the quadratic
		residuosity assumption (a simple modification of the Goldwasser-Micali
		cryptosystem suffices).
	\end{enumerate}

\end{enumerate}

% section Part II -- Security Definitions (end) ]]]}}}

\end{document}
